%%
% Plantilla de Memoria
% Modificación de una plantilla de Latex de Nicolas Diaz para adaptarla 
% al castellano y a las necesidades de escribir informática y matemáticas.
%
% Editada por: Mario Román
%
% License:
% CC BY-NC-SA 3.0 (http://creativecommons.org/licenses/by-nc-sa/3.0/)
%%

%%%%%%%%%%%%%%%%%%%%%
% Thin Sectioned Essay
% LaTeX Template
% Version 1.0 (3/8/13)
%
% This template has been downloaded from:
% http://www.LaTeXTemplates.com
%
% Original Author:
% Nicolas Diaz (nsdiaz@uc.cl) with extensive modifications by:
% Vel (vel@latextemplates.com)
%
% License:
% CC BY-NC-SA 3.0 (http://creativecommons.org/licenses/by-nc-sa/3.0/)
%
%%%%%%%%%%%%%%%%%%%%%

%----------------------------------------------------------------------------------------
%	PAQUETES Y CONFIGURACIÓN DEL DOCUMENTO
%----------------------------------------------------------------------------------------

%% Configuración del papel.
% microtype: Tipografía.
% mathpazo: Usa la fuente Palatino.
\documentclass[a4paper, 11pt]{article}
\usepackage[protrusion=true,expansion=true]{microtype}
\usepackage{mathpazo}

% Indentación de párrafos para Palatino
\setlength{\parindent}{0pt}
  \parskip=8pt
\linespread{1.05} % Change line spacing here, Palatino benefits from a slight increase by default


%% Castellano.
% noquoting: Permite uso de comillas no españolas.
% lcroman: Permite la enumeración con numerales romanos en minúscula.
% fontenc: Usa la fuente completa para que pueda copiarse correctamente del pdf.
\usepackage[spanish,es-noquoting,es-lcroman]{babel}
\usepackage[utf8]{inputenc}
\usepackage[T1]{fontenc}
\selectlanguage{spanish}


%% Gráficos
\usepackage{graphicx} % Required for including pictures
\usepackage{wrapfig} % Allows in-line images
\usepackage[usenames,dvipsnames]{color} % Coloring code


%% Matemáticas
\usepackage{amsmath}


%% Bibliografía
\makeatletter
\renewcommand\@biblabel[1]{\textbf{#1.}} % Change the square brackets for each bibliography item from '[1]' to '1.'
\renewcommand{\@listI}{\itemsep=0pt} % Reduce the space between items in the itemize and enumerate environments and the bibliography


% Hipervínculos
\usepackage[hidelinks]{hyperref}

%----------------------------------------------------------------------------------------
%	TÍTULO
%----------------------------------------------------------------------------------------
% Configuraciones para el título.
% El título no debe editarse aquí.
\renewcommand{\maketitle}{
  \begin{flushright} % Right align
  
  {\LARGE\@title} % Increase the font size of the title
  
  \vspace{50pt} % Some vertical space between the title and author name
  
  {\large\@author} % Author name
  \\\@date % Date
  \vspace{40pt} % Some vertical space between the author block and abstract
  \end{flushright}
}

% Título
\title{\textbf{Historia de las Matemáticas en China e India}\\ % Title
Aritmética, Resolución de Ecuaciones y Sistemas de Numeración} % Subtitle

\author{ \textsc{Óscar Bermúdez Garrido, \\
		Sonia González Sánchez,\\ 
		Alba María Toledo Pérez} % Authors
\\{\textit{Universidad de Granada}}} % Institution

\date{\today} % Date



%----------------------------------------------------------------------------------------
%	DOCUMENTO
%----------------------------------------------------------------------------------------

\begin{document}

\maketitle % Print the title section

% Resumen (Descomentar para usarlo)
\renewcommand{\abstractname}{Resumen} % Uncomment to change the name of the abstract to something else
%\begin{abstract}
% Resumen aquí
%\end{abstract}

% Palabras clave
%\hspace*{3,6mm}\textit{Keywords:} lorem , ipsum , dolor , sit amet , lectus % Keywords
%\vspace{30pt} % Some vertical space between the abstract and first section


% Índice
{\parskip=2pt
  \tableofcontents
}
\pagebreak

%% Inicio del documento

\section{Desarrollo de las Matemáticas en China}
	Aunque haya leyendas que ubican el comienzo de las matemáticas en la Antigua China, lo más correcto sería considerar
	que las matemáticas comenzaron en tiempos aún más remotos con métodos como las cuentas con nudos así como otros métodos
	utilizados en prácticamente todas las civilizaciones de la Antigüedad.
	
	Se tienen constancia del sistema de numeración decimal chino con alto grado de similitud al actual aproximadamente
	desde el siglo XIII a.C. siendo testigo de ello el uso de ábacos para calcular; sin embargo, hasta el siglo VII d.C.
	no se llegó a utilizar el cero. Además, dadas las diferencias lingüísticas y geográficas, el desarrollo de las matemáticas
	chinas se pueden considerar independientes de otras culturas como la griega, la egipcia,... A pesar de que debido a la
	expansión del Islam, las matemáticas chinas tuvieron gran influencia en las matemáticas occidentales, se consideraron
	independientes hasta el siglo XVII, momento en que los Nueve Capítulos sobre el Arte Matemático alcanzaron su forma final.
	
	Al igual que en otras sociedades primitivas, el desarrollo de las matemáticas se centró en la astronomía para perfeccionar
	el calendario agrícola y otras tareas prácticas como los negocios, la medida de las tierras, los impuestos,... pero
	la diferencia con estas sociedades se tiene en que el álgebra china tuvo su cenit en el siglo XIII, cuando Zhu Shijie
	inventó el método de cuatro desconocidos. Se tiene constancia de que los chinos desarrollaron cálculos con números
	muy grandes (sin duda, debido a la facilidad que brinda el empleo de un sistema decimal), números negativos, decimales,
	y un sistema binario. De igual manera, fueron capaces de obtener una demostración original del teorema de Pitágoras,
	conocían el triángulo de Pascal antes del nacimiento de Pascal, aproximaron el número $\pi$ y lograron resolver las
	ecuaciones de primer grado sobre el tablero de damas.
	
	\subsection{Ubicación y territorio}
	
	\subsection{Sistema de numeración chino y aritmética}

\section{Matemáticos destacados de China}
\section{Desarrollo de las Matemáticas en la India}
\section{Matemáticas destacados de la India}

\newpage
\addcontentsline{toc}{section}{Referencias}
\begin{thebibliography}{10}
\expandafter\ifx\csname url\endcsname\relax
  \def\url#1{\texttt{#1}}\fi
\expandafter\ifx\csname urlprefix\endcsname\relax\def\urlprefix{URL }\fi
\expandafter\ifx\csname href\endcsname\relax
  \def\href#1#2{#2} \def\path#1{#1}\fi

\bibitem{prezi_ch}
Aportaciones de los Chinos a las Matemáticas\\
  \url{https://prezi.com/5xzrxewmjjxo/aportaciones-de-los-chinos-a-las-matematicas/}

\bibitem{wiki_ch_mat}
Chinese Mathematics. Wikipedia, the free encyclopedia.\\
  \url{https://en.wikipedia.org/wiki/Chinese_mathematics}

\bibitem{sm_ch}
El Sistema de Numeración Chino. Sector Matemática.\\
  \url{http://www.sectormatematica.cl/historia/chino.htm}

\bibitem{uco_ch}
Sistema de Numeración Multiplicativo de China. Universidad de Córdoba.\\
  \url{http://www.uco.es/users/ma1fegan/2010-2011/md/Temas/Tema-1/Sistema-de-numeracion-China.pdf}

\bibitem{wiki_suanpan}
Suanpan. Wikipedia, the free encyclopedia.\\
  \url{https://en.wikipedia.org/wiki/Suanpan}
  
\bibitem{wiki_JiuzhangSuanshu}
The Nine Chapters on the Mathematical Art. Wikipedia, the free encyclopedia.\\
  \url{https://en.wikipedia.org/wiki/The_Nine_Chapters_on_the_Mathematical_Art}  
  
\bibitem{mp_LiuHui}
LIU HUI. Matematicas Proyecto.\\
  \url{http://trabajodematematicascambridgeschool.blogspot.com.es/2014/03/liu-hui.html}

\bibitem{mi_LiuHui}
Liu Hui. math.info\\
  \url{http://apprendre-math.info/espagnol/historyDetail.htm?id=Liu_Hui}
  
\bibitem{wiki_LiuHui}
Liu Hui. Wikipedia, the free encyclopedia.\\
  \url{https://en.wikipedia.org/wiki/Liu_Hui}

\end{thebibliography}


\end{document}
