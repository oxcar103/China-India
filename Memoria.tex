%%
% Plantilla de Memoria
% Modificación de una plantilla de Latex de Nicolas Diaz para adaptarla 
% al castellano y a las necesidades de escribir informática y matemáticas.
%
% Editada por: Mario Román
%
% License:
% CC BY-NC-SA 3.0 (http://creativecommons.org/licenses/by-nc-sa/3.0/)
%%

%%%%%%%%%%%%%%%%%%%%%
% Thin Sectioned Essay
% LaTeX Template
% Version 1.0 (3/8/13)
%
% This template has been downloaded from:
% http://www.LaTeXTemplates.com
%
% Original Author:
% Nicolas Diaz (nsdiaz@uc.cl) with extensive modifications by:
% Vel (vel@latextemplates.com)
%
% License:
% CC BY-NC-SA 3.0 (http://creativecommons.org/licenses/by-nc-sa/3.0/)
%
%%%%%%%%%%%%%%%%%%%%%

%----------------------------------------------------------------------------------------
%	PAQUETES Y CONFIGURACIÓN DEL DOCUMENTO
%----------------------------------------------------------------------------------------

%% Configuración del papel.
% microtype: Tipografía.
% mathpazo: Usa la fuente Palatino.
\documentclass[a4paper, 11pt]{article}
\usepackage[protrusion=true,expansion=true]{microtype}
\usepackage{mathpazo}

% Indentación de párrafos para Palatino
\setlength{\parindent}{0pt}
  \parskip=8pt
\linespread{1.05} % Change line spacing here, Palatino benefits from a slight increase by default


%% Castellano.
% noquoting: Permite uso de comillas no españolas.
% lcroman: Permite la enumeración con numerales romanos en minúscula.
% fontenc: Usa la fuente completa para que pueda copiarse correctamente del pdf.
\usepackage[spanish,es-noquoting,es-lcroman]{babel}
\usepackage[utf8]{inputenc}
\usepackage[T1]{fontenc}
\selectlanguage{spanish}


%% Gráficos
\usepackage{graphics, graphicx} % Required for including pictures
\usepackage{wrapfig} % Allows in-line images
\usepackage[usenames,dvipsnames]{color} % Coloring code


%% Matemáticas
\usepackage{amsmath}


%% Bibliografía
\makeatletter
\renewcommand\@biblabel[1]{\textbf{#1.}} % Change the square brackets for each bibliography item from '[1]' to '1.'
\renewcommand{\@listI}{\itemsep=0pt} % Reduce the space between items in the itemize and enumerate environments and the bibliography


% Hipervínculos
\usepackage[hidelinks]{hyperref}

%----------------------------------------------------------------------------------------
%	TÍTULO
%----------------------------------------------------------------------------------------
% Configuraciones para el título.
% El título no debe editarse aquí.
\renewcommand{\maketitle}{
  \begin{flushright} % Right align
  
  {\LARGE\@title} % Increase the font size of the title
  
  \vspace{50pt} % Some vertical space between the title and author name
  
  {\large\@author} % Author name
  \\\@date % Date
  \vspace{40pt} % Some vertical space between the author block and abstract
  \end{flushright}
}

% Título
\title{\textbf{Historia de las Matemáticas en China e India}\\ % Title
Aritmética, Resolución de Ecuaciones y Sistemas de Numeración} % Subtitle

\author{ \textsc{Óscar Bermúdez Garrido, \\
		Sonia González Sánchez,\\ 
		Alba María Toledo Pérez} % Authors
\\{\textit{Universidad de Granada}}} % Institution

\date{\today} % Date



%----------------------------------------------------------------------------------------
%	DOCUMENTO
%----------------------------------------------------------------------------------------

\begin{document}

\maketitle % Print the title section

% Resumen (Descomentar para usarlo)
\renewcommand{\abstractname}{Resumen} % Uncomment to change the name of the abstract to something else
%\begin{abstract}
% Resumen aquí
%\end{abstract}

% Palabras clave
%\hspace*{3,6mm}\textit{Keywords:} lorem , ipsum , dolor , sit amet , lectus % Keywords
%\vspace{30pt} % Some vertical space between the abstract and first section


% Índice
{\parskip=2pt
  \tableofcontents
}
\pagebreak

%% Inicio del documento

\section{Desarrollo de las Matemáticas en China}
	Aunque haya leyendas que ubican el comienzo de las matemáticas en la Antigua China, lo más correcto sería considerar
	que las matemáticas comenzaron en tiempos aún más remotos con métodos como las cuentas con nudos así como otros métodos
	utilizados en prácticamente todas las civilizaciones de la Antigüedad.
	
	Se tienen constancia del sistema de numeración decimal chino con alto grado de similitud al actual aproximadamente
	desde el siglo XIII a.C. siendo testigo de ello el uso de ábacos para calcular; sin embargo, hasta el siglo VII d.C.
	no se llegó a utilizar el cero. Además, dadas las diferencias lingüísticas y geográficas, el desarrollo de las matemáticas
	chinas se pueden considerar independientes de otras culturas como la griega, la egipcia,... A pesar de que debido a la
	expansión del Islam, las matemáticas chinas tuvieron gran influencia en las matemáticas occidentales, se consideraron
	independientes hasta el siglo XVII, momento en que los Nueve Capítulos sobre el Arte Matemático alcanzaron su forma final.
	
	Al igual que en otras sociedades primitivas, el desarrollo de las matemáticas se centró en la astronomía para perfeccionar
	el calendario agrícola y otras tareas prácticas como los negocios, la medida de las tierras, los impuestos,... pero
	la diferencia con estas sociedades se tiene en que el álgebra china tuvo su cenit en el siglo XIII, cuando Zhu Shijie
	inventó el método de cuatro desconocidos. Se tiene constancia de que los chinos desarrollaron cálculos con números
	muy grandes (sin duda, debido a la facilidad que brinda el empleo de un sistema decimal), números negativos, decimales,
	y un sistema binario. De igual manera, fueron capaces de obtener una demostración original del teorema de Pitágoras,
	conocían el triángulo de Pascal antes del nacimiento de Pascal, aproximaron el número $\pi$ y lograron resolver las
	ecuaciones de primer grado sobre el tablero de damas.
	
	\subsection{Sistema de numeración chino y aritmética}
		En torno al siglo IV a.C., aparecieron las primeras cifras talladas en caparazones de tortugas y huesos de animales,
		aunque hasta el siglo IV d.C. no llegaron las cifras que se utilizan actualmente.
		
		Estas cifras vienen representadas mediante caracteres del lenguaje que podemos ver en la figura \ref{fig:ch_numbers}
		denotan las unidades del 1 al 9 así como las potencias de 10.
		
		\begin{figure}[!ht]
			\centering
			\includegraphics[width = 14cm]{chinese_numbers.jpg}
			\caption{Caracteres chinos utilizados en el sistema de numeración.}
			\label{fig:ch_numbers}
		\end{figure}
					
		El sistema de numeración utilizado era el multiplicativo o híbrido, es decir, en este sistema las cifras de las
		unidades multiplican a las cifras de las potencias de 10 y el resultado obtenido se suma, de igual manera al método
		utilizado al pronunciar un número. Además, en este sistema no se necesitaba cifra para el cero.
		
		Habría que hacer mención también a la forma de escritura, pues, a diferencia de la escritura de occidente, en
		China se escribía tradicionalmente de arriba a abajo aunque también se hacía de izquierda a derecha. Cabe destacar
		que para evitar falsificaciones y errores en los documentos importantes se usaba una grafía algo más complicada.
	
		Para la realización de las operaciones habituales de la aritméticas en lugar de las cifras que hemos descrito
		anteriormente, se hacía uso del ábaco chino o \textit{suan pan}. Podemos ver un ejemplo de \textit{suan pan} en
		la imagen \ref{fig:ch_abacus}. Cada columna representa el valor de una potencia de 10, siendo la posición
		de las unidades la columna más a la derecha. Además, para cada columna disponemos de 2 cuentas de cielo que nos
		indican si el número es mayor que 5 o menor y 5 cuentas de tierra para expresar exactamente qué número es.
		Aunque pueda parecer innecesario el uso de 2 cuentas de cielo pues basta con una para expresar el número, esta
		cuenta extra se utilizaba para realizar con mayor facilidad las cuentas de multiplicación y división de números.
		
		\begin{figure}[!ht]
			\centering
			\includegraphics[width = 14cm]{Chinese-abacus.jpg}
			\caption{Ábaco \textit{suan pan} utilizado en China desde la Antigüedad.}
			\label{fig:ch_abacus}
		\end{figure}

	\subsection{\textit{Jiuzhang Suanshu} o \textit{Nueve Capítulos sobre el Arte Matemático}}
		\textit{Jiuzhang Suanshu} es un libro de matemáticas chino anónimo, y cuyos orígenes no son claros, compuesto por
		varias generaciones de eruditos del siglo X a.C.-II d.C., siendo además uno de los textos matemáticos más antiguos
		de China.
		
		En contraparte con con los matemáticos griegos antiguos, que tendieron a deducir proposiciones a partir de un
		conjunto inicial de axiomas, esta obra se centra en encontrar los métodos más generales de resolver problemas.
		Por ello, las entradas en el libro toman generalmente la forma de una declaración de un problema, seguida por
		la declaración de la solución, y una explicación del procedimiento que condujo a la solución.
				
		La influencia de este libro ayudó grandemente al desarrollo de matemáticas antiguas en las regiones de Corea y de
		Japón. Su influencia en el pensamiento matemático en China persistió hasta la era de la dinastía Qing(1644-1912).
		
		El contenido de \textit{Jiuzhang Suanshu} es:
		\begin{itemize}
			\item \textit{Fangtian} - Campos delimitadores. Áreas de campos de diversas formas, manipulación de fracciones vulgares.
			\item \textit{Sumi} - Mijo y arroz. Intercambio de productos a diferentes ritmos. Precios
			\item \textit{Cuifen} - Distribución proporcional. Distribución de materias primas y dinero a tasas proporcionales. Derivando en sumas aritméticas y geométricas.
			\item \textit{Shaoguang} - Reducción de dimensiones. División por números mixtos. Extracción de raíces cuadradas y cubicas. Diámetro de la esfera, perímetro y diámetro del círculo.
			\item \textit{Shanggong} - Para la construcción. Volúmenes de sólidos de diversas formas.
			\item \textit{Junshu} - Impuestos equitativos. Problemas más avanzados en la proporción.
			\item \textit{Yingbuzu} - Exceso y déficit. Los problemas lineales resueltos usando el principio conocido más tarde en Occidente como la regla de la falsa posición.
			\item \textit{Fangcheng} - La referencia a dos caras (es decir, las ecuaciones). Sistemas de ecuaciones lineales, resueltos por un principio similar a la eliminación gaussiana.
			\item \textit{Gougu} - Base y altitud. Problemas que implican el teorema de Gougu, un teorema idéntico al de Pitágoras.
		\end{itemize}
		
\section{Matemáticos destacados de China}

	\subsection{Liu Hui (220 - 280)}
		Liu Hui fue un matemático chino que vivió en el reinado de Wei en el período de los Tres Reinos en el que se
		fraguaron dos guerras. Además, junto con Zu Chongzhi fue conocido como uno de los matemáticos más grandes de la
		Antigua China.

		Como gran aportación al mundo de las matemáticas cabe destacar un comentario realizado a \textit{Jiuzhang Suanshu}
		o \textit{Nueve Capítulos sobre el Arte Matemático} que se publicó en el año 263. En este comentario, destacan:
		
		\begin{itemize}
			\item La explicación de una demostración al teorema de Gougu.
			\item Encontró una relación por recurrencia para expresar la longitud del lado de un polígono regular con
			$3 \cdot 2^n$ lados en términos de la longitud del lado de un polígono regular con $3 \cdot 2^{n-1}$ lados.
			Esto se logra con una aplicación del teorema de Gougu.
			\item Un algoritmo para el cálculo de $\pi$. Calculó que 3,141024 < $\pi$ < 3,142074 utilizando para ello un
			polígono de 192 (= $3 \cdot 64$) lados mientras que Arquímedes utilizó un polígono 96 lados para obtener
			$\frac {223}{71} < \pi < \frac{22}{7}$. De esta manera, los resultados de Liu Hui resultaron algo más precisos
			que los de Arquímedes. Como 3.142074 le pareció demasiado grande, escogió los tres primeros dígitos, es decir,
			3.14. Posteriormente, mejoró el método y obtuvo $\pi = 3.1416$ con un polígono de 3072(= $3 \cdot 1024$) lados.
			En yuxtaposición, en el \textit{Jiuzhang Suanshu} habían utilizado el valor 3 para $\pi$, aunque también se
			había estimado previamente $\pi = \sqrt{10}$.
			\item La solución de sistemas de ecuaciones lineales a través del método que conocemos como eliminación de Gauss.
			\item El cálculo del volumen del prisma, el tetraedro, la pirámide, el cilindro, el cono y el tronco cónico,
			siendo así uno de los mayores contribuyentes a la geometría sólida empírica.
			\item El principio de Cavalieri para encontrar el volumen de un cilindro y la intersección de dos cilindros
			perpendiculares aunque este trabajo fue terminado solamente por  Zu Chongzhi y Zu Gengzhi.
		\end{itemize}

		También escribió ese mismo año el \textit{Haidao suajing} o \textit{Manual Matemático de la Isla Marina} que es
		básicamente un manual práctico de las matemáticas con el objetivo proporcionar métodos para resolver problemas
		cotidianos de la ingeniería, topografía, el comercio y la fiscalidad. Entre los problemas a los que plantea
		solución se encuentran:
		
		\begin{itemize}
			\item La medida de la altura de una isla opuesta a su nivel del mar y vista desde el mar.
			\item La altura de un árbol en una colina.
			\item El tamaño de un muro de la ciudad visto a larga distancia.
			\item La profundidad de un barranco.
			\item La altura de una torre en una llanura vista desde una colina.
			\item La anchura de una boca de río vista desde una distancia en tierra.
			\item El ancho de un valle visto desde un acantilado.
			\item La profundidad de una piscina transparente.
			\item El ancho de un río visto desde una colina.
			\item El tamaño de una ciudad vista desde una montaña.
		\end{itemize}

	\subsection{Zu chongzhi (429 - 501)}
		Zu Chongzhi (429-501 d.C.) fue un destacado matemático y astrónomo chino nacido en Jiankang (hoy Nanjing) que
		vivió durante las dinastías de Liu Song y Qi del Sur. Su abuelo, Zu Chang ocupó el cargo de Ministro Principal
		de los Edificios del Palacio durante la dinastía de Liu Song y estuvo a cargo de los proyectos de construcción
		del gobierno. Su padre, Zu Shuozhi también sirvió a la corte y fue muy respetado por su sabiduría. Su familia
		había estado históricamente implicada en la investigación de la astronomía, y desde su niñez estuvo en contacto
		con matemáticos y astrónomos. Debido a su talento desde temprana edad, el emperador Xiaowu de Liu Song oyó hablar
		de él y lo envió a una Academia, donde amplió sus conocimientos matemáticos con los comentarios que Liu Hui
		escribió sobre el \textit{Jiuzhang Suanshu}, y más tarde fue a la Universidad Imperial de Nanjing. En 461 d.C.,
		en Nanxu (hoy Zhenjiang, Jiangsu), ocupó un trabajo en la oficina del gobernador local.
		
		Sus logros matemáticos incluyeron\footnote{Representaremos entre paréntesis los valores que se toman por ciertos
		en la actualidad, para así valorar la bondad de sus aproximaciones.}:
		\begin{itemize}
			\item Distinguir el año sideral y el año tropical, y medir 45,9 ($\approx$ 70,7) años por grado entre ellos.
			\item Calcular un año como 365.24281481($\approx$ 365.24219878) días.
			\item Calcular el número de solapamientos entre el Sol y la Luna como 27.21223($\approx$ 27.21222). Además,
			utilizó este número, para predecir con éxito cuatro eclipses.
			\item Calcular el año de Júpiter como 11.858($\approx$ 11.862) años terrestres.
			\item Introducir el calendario Daming en 465 d.C. utilizando los valores expresados anteriormente.
			\item Acotó la constante $\pi$ entre 3.1415926 y 3.1415927, de forma similar a como hizo Liu, usando un
			polígono de 24.576 (= $3 \cdot 2^13$) lados. Expresó además la fracción de Zu ($\frac{355}{113}$) como
			aproximación a $\pi$ y resultó ser la aproximación racional más cercana a $\pi$ de todas las fracciones con
			denominador inferior a 16600. Por desgracia para Occidente, esta aproximación tardó más de 900 años en llegar
			hasta Europa.
			\item Deducir el volumen de una esfera como $\frac{\pi D^3}{6}$ donde $D$ es diámetro para Oriente, ya que
			a pesar de que Arquímedes ya había descubierto la fórmula equivalente $\frac{4 \pi r^3}{3}$ con $r$ el radio
			para Occidente, las diferenciales culturales y geográficas aislaron a China de este descubrimiento.
		\end{itemize}
		
		Gran parte de estos logros los recogió en un libro matemático llamado \textit{Zhui Shu} o \textit{Métodos para la
		Interpolación} que escribió en colaboración con su hijo Zu Geng, quien también fue un matemático sobresaliente.
		Desgraciadamente, sus métodos de interpolación y el uso de integración eran muy adelantados para su época, resultó
		muy confuso para los estudiantes de la Academia Imperial y fue descartado del programa y olvidado, por lo que se
		perdió desde la dinastía Song.
	
	\subsection{Qin Jiushao (1202 - 1261)}
		Qin Jiushao nació en Ziyang, y es considerado como uno de los más grandes matemáticos del siglo XIII, así como
		de la historia china. Hecho especialmente asombroso ya que no dedicó su vida a las matemáticas. Qin era experto
		en esgrima, tiro con arco, montar a caballo, la música y la arquitectura y tuvo una serie de puestos administrativos
		en las burocracias de varias provincias chinas.
		
		En 1219, se ofreció voluntariamente para el ejército y sirvió durante un tiempo. Como funcionario del gobierno
		era jactancioso, corrupto, y fue acusado de soborno y de envenenar a sus enemigos, por ello, fue relevado de sus
		deberes varias veces pero logró hacerse muy rico.
		
		Al igual, que todo erudito chino, Qin Jiushao estudió el \textit{Jiuzhang Suanshu}. En 1247 d.C., escribió la
		obra por la cuál tiene su reputación matemática, el \textit{Shushu Jiuzhang} o \textit{Tratado Matemático en Nueve
		Secciones} que sigue, en cierta medida, el modelo del \textit{Jiuzhang Suanshu}, aunque de forma más sofisticada.
		
		Este libro es el primer ejemplar que encontramos que utiliza \textit{0} para el cero\footnote{En todos los libros
		antiguos nos encontramos con lugares vacíos.} y se caracteriza por estar compuesto por 9 capítulos con 9 problemas
		prácticos cada uno, como problemas de calendario, musicales y fiscales. Estos capítulos eran:
		\begin{itemize}
			\item Ecuaciones indeterminadas.
			\item Fenómenos celestiales.
			\item Área de tierra y campo.
			\item Topografía.
			\item Impuestos.
			\item Almacenamiento de granos.
			\item Construcción de edificio.
			\item Asuntos militares.
			\item Precio e interés.
		\end{itemize}
		
		En este tratado, se describe el Teorema Chino del Resto junto con una demostración constructiva, re-descubrió
		para Oriente "la fórmula de Qin Jiushao" para calcular el área de un triángulo dadas las longitudes de sus
		lados\footnote{En Occidente fue conocida como fórmula de Herón desde el año 60 aC, aunque podría deberse a
		Arquímedes.}, y obtuvo el "Método de Lin Long" para la solución numérica de ecuaciones polinómicas (equivalente
		al método de Horner del siglo XIX) y para encontrar sumas de series aritméticas.
	
	\subsection{Shiing-Shen Chern (1911 - 2004)}
		Shiing-Shen Chern(1911-2004) fue un matemático chino estadounidense considerado uno de los líderes en geometría
		diferencial del siglo XX.
		
		Nació en Jiaxing pero en 1922 se trasladó a Tianjin para estar con su padre. A partir de 1926 estudió en la
		Universidad de Nankai, graduándose en matemáticas en 1930. Estudió la geometría diferencial proyectiva de mano
		del profesor Sun Guangyuan, considerado un fundador de la matemática china moderna. En 1932, Chern publicó su
		primer artículo de investigación y Wilhelm Blaschke quedó impresionado por Chern y su investigación. En 1934,
		Chern se graduó y fue co-financiado a continuar su estudio en matemáticas en Alemania con una beca. En 1936, se
		doctoró bajo la guía de Blaschke, quién le recomendó que estudiara en París y así fue cómo trabajó con Élie Cartan.
		
		Tras un año en París, y volvió a enseñar matemáticas en Tsinghua. En 1943, Chern marcha al Instituto de Estudios
		Avanzados (IAS) de Princeton, trabajando allí sobre las clases características de geometría diferencial. Regresó
		a Shanghai en 1946 para fundar el Instituto de Matemáticas de la Academia Sínica, que más tarde fue trasladado a
		Nanking. A partir de 1948 fue de nuevo al IAS, convirtiéndose en profesor de la Universidad de Chicago en 1949.
		Se trasladó a la Universidad de California, Berkeley en 1960. Poco después, se hizo ciudadano estadounidense y
		fue elegido miembro de la Academia Nacional de Ciencias de los Estados Unidos. En 1964, Chern fue vicepresidente
		de la American Mathematical Society (AMS). Fue profesor en la Universidad de Chicago hasta que se retiró en 1979.
		 
		En Berkeley fundó el Instituto de Investigación de las Ciencias Matemáticas (MSRI) en 1981 y actuó como director
		hasta 1984. En 1985, fundó el Instituto de Matemáticas de Nankai(NKIM) en Tianjin. Éste fue renombrado en 2004
		como el Instituto de Chern de Matemáticas como homenaje tras la muerte de Chern.
		
		Como primer punto a destacar en la labor de Chern, se decía que sus cursos de grado de geometría diferencial en
		Chicago y Berkeley eran magníficos. Por otra parte, el trabajo de Chern abarca todos los campos clásicos de la
		geometría diferencial:
		\begin{itemize}
			\item La teoría Chern-Simons
			\item La teoría Chern-Weil
			\item Las clases de Chern
			\item La geometría diferencial proyectiva y redes matemáticas.
		\end{itemize}
		
		Ha publicado los resultados en la geometría integral, el valor de distribución de la teoría de funciones
		holomórficas, y superficies mínimas. Fue seguidor de Élie Cartan, trabajando intensamente en la \textit{teoría
		de la equivalencia} a su vez en China de 1937 a 1943, en relativo aislamiento. En 1954, publicó su propio
		tratamiento del problema de pseudogrupo, que es en efecto una gran base de la teoría geométrica de Cartan.
		De hecho, uno de sus libros se titula \textit{Complex Manifolds without Potential Theory}.
		
		Se le concedió la Medalla Nacional de Ciencias en 1975, el premio Wolf en matemáticas en 1984, y el premio Shaw
		en ciencias matemáticas en mayo de 2004. En su honor, cada cuatro años el Congreso Internacional de Matemáticos y
		la Unión Matemática Internacional entregan la Medalla Chern, con la que se premia la dedicación de toda una vida
		de trabajo a las matemáticas en su nivel más alto.
			
	\subsection{Chen Jingrun (1933 - 1996)}
		Chen Jingrun, fue un matemático chino que hizo contribuciones significativas a la teoría numérica. Nació en 1933
		y estudió en la Universidad de Xiamen, graduándose en 1953.
		
		Trabajó sobre la conjetura principal gemela, el problema de Waring, la conjetura de Legendre y la conjetura de
		Goldbach consiguiendo así un gran progreso en la teoría numérica analítica.
		
		En 1966, demostró lo que se conoce actualmente como una de las mejores aproximaciones a la conjetura de Goldbach,
		recibe el nombre de \textit{Teorema de Chen}:
		
		\begin{center}
			\em
			Todo entero par suficientemente grande se puede escribir como la suma de un número primo y de un semiprimo
			o casiprimo, es decir, un entero que es el producto de, a lo sumo, dos números primos.
		\end{center} 
		
		Fue un ídolo nacional en la década de los setenta.
				
\section{Desarrollo de las Matemáticas en la India}
	La matemática india se desarrolló desde el 1200 a.C. hasta el siglo XVIII. Al igual que China, este desarrollo se
	considera prácticamente independiente del desarrollo matemático originado en Europa, aunque tuvieron algunos
	puntos de encuentro, como por ejemplo:
	\begin{itemize}
		\item La marcha de Alejandro Magno sobre la India durante el siglo IV a.C., que produjo un intercambio de ideas
		con el mundo griego.
		\item La expansión del budismo en China.
		\item La expansión del mundo árabe, que dio a conocer el sistema posicional decimal con uso del 0 a Europa, que
		dieron un impulso al desarrollo de las matemáticas hasta el punto de que es el sistema que se usa actualmente
		en todo el mundo.
	\end{itemize}	

	Se puede apreciar dos épocas de desarrollo en las matemáticas indias, siendo:
	\begin{itemize}
		\item La época clásica(siglos I al VIII): En esta época se llevaron a cabo grandes contribuciones de la mano de
		matemáticos como Aryabhata, Varahamihira, Brahmagupta, Bhaskara I, Mahavira, Bakhshali, Bhaskara II, Madhava de
		Sangamagrama y Nilakantha Somayaji.
		\item La escuela de Kerala(1300- 1600): en esta escuela se realizaron grandes desarrollos matemáticos, aunque
		por otra parte, desgraciadamente no parece haber indicios de difusión de estos conocimientos fuera de la escuela.
	\end{itemize}

	A diferencia del desarrollo de las matemáticas en Occidente y en similitud con las matemáticas chinas, las matemáticas
	indias dieron preferencia al cálculo numérico frente al rigor deductivo.

	Las primeras obras indias se transmitían de forma oral(y ésta fue la única forma hasta el 500 a.C.) y mediante la
	escritura en manuscritos. Debido a esta transmisión oral, los documentos solían componerse de una parte en verso
	para facilitar el aprendizaje donde usualmente se planteaba el problema y de una segunda parte en prosa comentando
	detalladamente la explicación del problema así como su solución.
	
	Entre sus mayores logros, se pueden encontrar:
	\begin{itemize}
		\item El estudio del concepto de cero como número.
		\item Las primeras cuentas con números negativos para representar deudas.
		\item El planteamiento de forma algebraica de la teoría de ecuaciones de primer y segundo grado.
		\item El descubrimiento de algunas ternas pitagóricas.
		\item El cálculo de raíces cuadradas.
		\item La primera aparición de la lógica Booleana y las gramáticas libres de contexto
		\item El desarrollo de las definiciones de seno, coseno y otros conceptos básicos de la trigonometría.
		\item El desarrollo en la escuela de Kerala de las expansiones en serie para las funciones trigonométricas siendo
		éste el primer ejemplo de una serie de potencias. Desgraciadamente, no formularon una teoría sistemática de
		diferenciación e integración.
	\end{itemize}

	\subsection{Sistema de numeración indio y aritmética}
		La mayor característica y sin duda la más conocida del sistema de numeración india es el uso de un sistema posicional
		de base 10 con cero, siendo pioneros en el uso de este sistema de numeración que está tan arraigado actualmente.
		Usaron además una grafía muy similar a la actual y que podemos ver en la figura \ref{fig:ind_numbers}:
		\begin{figure}[!ht]
			\centering
			\includegraphics[width = 14cm]{indian_numbers.jpg}
			\caption{Símbolos utilizados en el sistema de numeración indio.}
			\label{fig:ind_numbers}
		\end{figure}

		Una característica algo peculiar de este sistema fue que en lugar de tener un $"$.$"$ en la posición de los millares,
		conocido como \textit{punto de los millares}, utilizaban un $"$.$"$ en la posición de las centenas, esto es, para
		escribir un millón, ellos lo notaban \textit{1.00.00.00}.

		En la multiplicación hindú, también conocida como multiplicación de Fibonacci o multiplicación en celosía se
		dibujaba una tabla en la que se trazan las diagonales de izquierda a derecha que sirve de guía para el cálculo:
		\begin{enumerate}
			\item El multiplicando se escriben encima de la tabla.
			\item El multiplicador se escribe a la derecha de la tabla.
			\item La tabla se rellena con los productos de los dígitos que señalan cada fila y columna: las decenas se
			escriben en la esquina superior izquierda de cada celda, y las unidades en la inferior derecha.
			\item Se suma la tabla según las diagonales.
			\item Si es necesario "llevarse" las decenas, se muestra la solución de arriba abajo y de izquierda a derecha
			del borde de la tabla, llevándose las decenas en sentido inverso, como se haría de forma habitual.
		\end{enumerate}

		Para aclarar el algoritmo, se brinda la figura \ref{fig:ind_multiplication}:

		\begin{figure}[!ht]
			\centering
			\includegraphics[width = 14cm]{indian_multiplication.png}
			\caption{Multiplicación en celosía.}
			\label{fig:ind_multiplication}
		\end{figure}
		
		Para realizar la división utilizaba el método conocido como división por galera, que es un antiguo algoritmo de
		división, utilizado de manera habitual hasta el siglo XVII. El nombre deriva del parecido gráfico que se genera
		con este método y una galera. No se puede llegar a confirmar pero se cree que el origen de este método puede ser
		árabe o hindú o chino.
		
		El algoritmo de esta división es:
		\begin{enumerate}
			\item Escribir el dividendo y dibujar una barra vertical tras el dividendo en la que iremos escribiendo el
			cociente conforme se vaya calculando.
			\item Escribir el divisor bajo el dividendo de tal forma que los dígitos más a la izquierda coincidan.
			\item Suponer que el dividendo acaba donde acaba el divisor y realizar la división.\label{division}
			\item Si el número es 0, tachamos el divisor y pasamos al siguiente paso; si no, seguimos el siguiente
			sub-algoritmo:
			\begin{enumerate}
				\item Multipliclar el primer dígito por el nuevo dígito del cociente y restar del primer dígito del
				dividendo, escribimos el resultado encima del dividendo y tachamos tanto el primer dígito del
				dividendo como el divisor.
				\item Operamos de igual forma con el resto de dígitos del divisor dígito a dígito, utilizando si nos
				fuese necesario los dígitos escritos encima del dividendo como nuevo dividendo y tachándolos de igual
				manera si varían.
			\end{enumerate}
			\item Volvemos a escribir el cociente debajo del último cociente pero desplazado una posición a la derecha.
			\item Si el divisor excede por la derecha al dividendo, los números que no están tachados son el resto; si
			no, volvemos al paso \ref{division} tomando como nuevo dividendo los números que no están tachados.
		\end{enumerate}
		
		El método de la galera, aunque más complejo, es similar al método moderno de la división larga. Dado que el
		algoritmo puede llegar a ser algo confuso, por ello, he recogido tanto el proceso paso a paso en la figura
		\ref{fig:ind_division} como el resultado final, en el cuál, los números que no aparecen tachados en la figura
		\ref{fig:ind_division2} son el resto y el cociente.

		\begin{figure}[!ht]
			\centering
			\includegraphics[width = 10cm]{Galley_Method3.png}
			\caption{División en galera paso a paso de 65284 entre 594.}
			\label{fig:ind_division}
		\end{figure}

		\begin{figure}[!ht]
			\centering
			\includegraphics[width = 7cm]{Galley_Method4.png}
			\caption{Solución final de la división en galera de 65284 entre 594.}
			\label{fig:ind_division2}
		\end{figure}
		
\section{Matemáticas destacados de la India}
	\subsection{Aryabhata(476-550)}
		Aryabhata(476-550) fue el primer gran matemático y astrónomo de la era clásica de la matemática en la India.
		Nació en Taregana, cerca de Pataliputra. Se deduce que nació en el año 476 por datos de su principal obra,
		\textit{Aryabhatiya}, publicada en el 499. Por tanto, es el escritor de álgebra más antiguo de quien se tiene
		conocimiento.
		
		Viajó a Pataliputra para estudiar y vivió ahí por un tiempo. Posiblemente fue el jefe de una Universidad en
		Pataliputra y su observatorio astronómico. También se cree que estableció un observatorio en el templo del Sol
		en Taregana. Fue mentor de Bhaskara I, otro gran matemático.
				
		Aryabhata fue autor de varios tratados en matemáticas y astronomía, algunos se perdieron. Sin embargo, llegó
		hasta nuestros días una obra llamada \textit{Aryabhatiya}, nombre que posiblemente no le diera Aryabhata sino
		sus comentaristas posteriores. Esta obra trata principalmente sobre la matemática y la astronomía, siendo un
		compendio de los conocimientos de la época y teniendo gran parte de procedencia griega. Tuvo una gran influencia
		en la matemática de la India.

		\textit{Aryabhatiya} consiste en 108 versos y 13 versos introductorios, y está dividido en cuatro capítulos:
		\begin{itemize}
			\item \textit{Gitikapada}: Tiene 13 versos que tratan grandes unidades de tiempo, los cuales presentan una
			cosmología diferente a textos anteriores. Hay también una tabla de senos, dada en un único verso.
			\item \textit{Ganitapada}: Tiene 33 versos cubriendo medición, aritmética y progresiones geométricas, sombras,
			ecuaciones simples, cuadráticas, simultáneas, e indeterminadas.
			\item \textit{Kalakriyapada}: Tiene 25 versos que presentan diferentes unidades de tiempo y un método para
			determinar las posiciones de los planetas para un día dado, cálculos relacionados con el mes bisiesto y una
			semana de siete días con nombres para los días de la semana.
			\item \textit{Golapada}: Tiene 50 versos que plantean aspectos geométricos/trigonométricos de la esfera
			celeste, características de la eclíptica, el ecuador celeste, nodo, forma de la Tierra, la causa del día
			y la noche, la subida de los signos zodiacales en el horizonte,...
		\end{itemize}
		
		El carácter en el que \textit{Aryabhatiya} está escrito es meramente mnemotécnio, haciendo así que sólo haya
		podido ser descifrada a través de sus comentaristas como su discípulo Bhaskara I en el 600 o como Nilakantha
		Somayaji en 1465.
		
		En su obra, Aryabhata destacó por:
		\begin{itemize}
			\item Aproximar el número $\pi$ (3,1416) y se especula que indica que no solo es una aproximación sino
			que el valor es inconmensurable (o irracional), algo que no fue probado hasta 1761.
			\item Ser el primero en encontrar el radio de la Tierra y el único en tiempos antiguos de calcular su
			volumen.
			\item A pesar de no utilizar un símbolo para el cero, se cree que su conocimiento estaba implícito en su
			sistema de notación posicional como un marcador de posición para las potencias de 10 de coeficientes nulos.
			Por esto, algunos le consideran el padre de la numeración decimal.					
			\item Enunciar que \textit{para un triángulo, el resultado de una perpendicular con el semi-lado es el área}.
			\item Dar la suma de series matemáticas de cuadrados y cubos:
			\begin{itemize}
				\item $1^2 + 2^2 + \cdots + n^2 = \frac{n(n + 1)(2n + 1)}{6}$
				\item $1^{3} + 2^{3} + \cdots + n^{3} = (1+2+\cdots +n)^{2}$ 
			\end{itemize}
			\item Conocer la resolución de la ecuación de segundo grado, que algunos consideran su descubrimiento.
			\item Utilizó el método \textit{kuttaka}(\textit{pulverizar} o \textit{romper en pequeñas piezas}), que es
			un algoritmo recursivo para escribir los factores originales en números más pequeños, para encontrar
			soluciones enteras a ecuaciones diofánticas $ax + by = c$. Este problema fue de gran interés para los
			matemáticos indios y lo siguen siendo en la actualidad por sus aplicaciones en criptología. De hecho,
			actualmente este algoritmo es el método estándar para resolver ecuaciones diofánticas de primer orden y es
			a veces referido como el algoritmo de Aryabhata.
			\item Otros resultados de aritmética, álgebra, geometría, trigonometría plana y esférica, fracciones
			continuadas, ecuaciones cuadráticas y sumas de series de potencias.
		\end{itemize} 		

	\subsection{Varahamihira(505-587)}
		Varahamihira (505-587) fue un astrónomo, matemático y astrólogo hinduista que vivió en Ujjain. Se conoce poco
		acerca de su vida. Trabajó en la escuela de matemáticas en Ujjain, que había sido un centro importante para
		las matemáticas desde alrededor del año 400 d.C. y su importancia siguió creciendo dado que Varahamihira trabajó
		allí y Brahmagupta tiempo después. Varahamihira presentaba sus propias observaciones embellecidas con atractiva
		poesía y una gran variedad de estilos métricos.
		
		Varahamihira logró realizar grandes descubrimientos matemáticos:
		\begin{itemize}
			\item Encontró fórmulas trigonométricas equivalentes a:
					\begin{itemize}
						\item $sin x = cos( \frac{\pi}{2} - x)$
						\item $sin^2 x + cos^2 x = 1$
						\item $\frac{1 - cos(2x)}{2} = sin^2 x$
					\end{itemize}
			\item Obtuvo tablas de senos más precisas que las de Aryabhata. Estas mejoras de la exactitud eran de vital
			importancia para los matemáticos indios ya que se utilizaban posteriormente para aplicaciones en astronomía
			y astrología.
			\item Trabajó también en cuadrados mágicos, entre ellos, un cuadrado mágico pandiagonal de orden cuatro.
			\item Calculó el número de formas en que $r$ objetos se pueden seleccionar de $n$ objetos($_n C_r$) usando
			el triángulo de Pascal visto desde un ángulo diferente a como lo vemos hoy.
		\end{itemize}

		Su obra \textit{Brihatsamhita}(\textit{La Gran Compilación}) discute temas como:
		\begin{itemize}
			\item Descripciones de los cuerpos celestes, sus movimientos y conjunciones.
			\item Los fenómenos meteorológicos, las indicaciones de los presagios que estos movimientos, conjunciones
			y fenómenos representan así como qué acción tomar y las operaciones a realizar.
		\end{itemize}
		
		Pero su obra más famosa fue escrita en el 575 \textit{Pancasiddhantika}(\textit{Los Cinco Cánones Astronómicos}), 
		es importante por sí misma y por dar información sobre textos indios que se han perdido ya que es un tratado
		de astronomía matemática que resume cinco tratados astronómicos anteriores:
		\begin{enumerate}
			\item \textit{Surya siddhanta}.
			\item \textit{Romaka siddhanta}.
			\item \textit{Paulisa siddhanta}.
			\item \textit{Vasishtha siddhanta}.
			\item \textit{Paitamaha siddhanta}.
		\end{enumerate}

	\subsection{Brahmagupta (598-660)}
		Brahmagupta (598 - 670) fue un matemático y astrónomo indio y es considerado el mayor matemático de esta época.
		Posiblemente nació en Ujjain, donde vivió. Allí fue director del antiguo y famoso observatorio de astronomía de
		Uijain.
		
		Su principal obra, \textit{Brahmasphutasiddhanta}(\textit{La perfección de la obra de Brahma}), fue escrita en
		verso hacia el año 628. En ella, aparece por primera vez el concepto del cero y, dado que influyó notablemente
		en los matemáticos árabes, se cree que posiblemente Brahmagupta fue el idealizador de este concepto. También
		desarrolla un sistema de representación de fracciones y define sus operaciones básicas que es considerado el
		origen del actual. Trabaja además con progresiones aritméticas, diversos elementos de geometría, números
		negativos, eclipses, posiciones y conjunciones planetarias y fases lunares.

		Entre los mejores logros de Brahmagupta se tiene:
		\begin{itemize}
			\item Encontrar una regla para la formación de ternas pitagóricas: \\
			$m, {\frac{m^{2}}{2(m-n)}}, {\frac{m^{2}}{2(m+n)}}$
			\item La identidad de Brahmagupta, que enuncia que el producto de dos números, cada uno de los cuales es la
			suma de dos cuadrados, también es suma de dos cuadrados, es decir: \\
			$(a^2 + b^2)(c^2 + d^2) =(ac - bd)^2 + (ad + bc)^2 = (ac + bd)^2 + (ad-bc)^2$
			\item La fórmula de Brahmagupta, esta fórmula permite encontrar el área de cualquier cuadrilátero cíclico
			dadas las longitudes de los lados y algunos de los ángulos: \\
			$A = \sqrt{(s - a)(s - b)(s - c)(s - d)}$ donde $s$, es el semiperímetro $s = \frac {a + b + c + d}{2}$. 
			\item El teorema de Brahmagupta, que nos da una condición necesaria sobre la perpendicularidad de las
			diagonales de un cuadrilátero cíclico:
			\begin{center}
				\em Si las diagonales de un cuadrilátero cíclico son perpendiculares, entonces toda recta perpendicular
				a un lado cualquiera del cuadrilátero y que pase por la intersección de las diagonales, divide al lado
				opuesto en dos partes iguales.
			\end{center}
			\item Ser el primero en dar una solución general para la ecuación diofántica lineal $ax + by = c$ con $a,b,c
			\in \mathbb {Z}$.
			
			Además, Brahmagupta sabía que si $a$ y $b$ son primos entre sí, entonces todas las soluciones de la ecuación
			vienen dadas por las fórmulas: $x = p + m b$, $y = q - m a$ donde $m$ es un entero arbitrario.
		\end{itemize}
		
	    
\newpage
\addcontentsline{toc}{section}{Referencias}
\begin{thebibliography}{10}
\expandafter\ifx\csname url\endcsname\relax
  \def\url#1{\texttt{#1}}\fi
\expandafter\ifx\csname urlprefix\endcsname\relax\def\urlprefix{URL }\fi
\expandafter\ifx\csname href\endcsname\relax
  \def\href#1#2{#2} \def\path#1{#1}\fi

\bibitem{prezi_ch}
Aportaciones de los Chinos a las Matemáticas\\
  \url{https://prezi.com/5xzrxewmjjxo/aportaciones-de-los-chinos-a-las-matematicas/}

\bibitem{wiki_ch_mat}
Chinese Mathematics. Wikipedia, the free encyclopedia.\\
  \url{https://en.wikipedia.org/wiki/Chinese_mathematics}

\bibitem{sm_ch}
El Sistema de Numeración Chino. Sector Matemática.\\
  \url{http://www.sectormatematica.cl/historia/chino.htm}

\bibitem{uco_ch}
Sistema de Numeración Multiplicativo de China. Universidad de Córdoba.\\
  \url{http://www.uco.es/users/ma1fegan/2010-2011/md/Temas/Tema-1/Sistema-de-numeracion-China.pdf}

\bibitem{wiki_suanpan}
Suanpan. Wikipedia, the free encyclopedia.\\
  \url{https://en.wikipedia.org/wiki/Suanpan}
  
\bibitem{wiki_JiuzhangSuanshu}
The Nine Chapters on the Mathematical Art. Wikipedia, the free encyclopedia.\\
  \url{https://en.wikipedia.org/wiki/The_Nine_Chapters_on_the_Mathematical_Art}  
  
\bibitem{mp_LiuHui}
LIU HUI. Matematicas Proyecto.\\
  \url{http://trabajodematematicascambridgeschool.blogspot.com.es/2014/03/liu-hui.html}

\bibitem{mi_LiuHui}
Liu Hui. math.info\\
  \url{http://apprendre-math.info/espagnol/historyDetail.htm?id=Liu_Hui}
  
\bibitem{wiki_LiuHui}
Liu Hui. Wikipedia, the free encyclopedia.\\
  \url{https://en.wikipedia.org/wiki/Liu_Hui}

\end{thebibliography}


\end{document}
