%%
% Plantilla de Presentación
% Modificación de una plantilla de Latex de LaTeXTemplates para adaptarla 
% al castellano y a las necesidades de escribir informática y matemáticas.
%
% Editada por: Mario Román
%
% License:
% CC BY-NC-SA 3.0 (http://creativecommons.org/licenses/by-nc-sa/3.0/)
%%

%%%%%%%%%%%%%%%%%%%%%
% Beamer Presentation
% LaTeX Template
% Version 1.0 (10/11/12)
%
% This template has been downloaded from:
% http://www.LaTeXTemplates.com
%
% License:
% CC BY-NC-SA 3.0 (http://creativecommons.org/licenses/by-nc-sa/3.0/)
%
%%%%%%%%%%%%%%%%%%%%%

%----------------------------------------------------------------------------------------
%	PAQUETES Y CONFIGURACIÓN DEL DOCUMENTO
%----------------------------------------------------------------------------------------

\documentclass[12pt, aspectratio=169]{beamer} % Beamer
\usepackage[spanish]{babel} % Traducciones
\usepackage[utf8]{inputenc} % Uso de caracteres UTF-8
\usepackage[T1]{fontenc} % Permite copiar código y evita errores
\uselanguage{Spanish} % Traducciones beamer
\languagepath{Spanish} % (tex.stackexchange.com/questions/168208)
\usepackage{pgfpages} % Beamer User Guide sections 19.6 and 22
\usepackage[absolute,overlay]{textpos} % Especifica posición del texto.

%% Temas %%
% Tema y tema de color
\usetheme{Dresden}
\usecolortheme{dolphin}

% Fuentes de tamaño arbitrario
\usepackage{lmodern}

% Gráficos
\usepackage{caption}
\usepackage{subcaption} % Allows including 2 images side to side
\usepackage{graphicx} % Allows including images
\usepackage{booktabs} % Allows the use of \toprule, \midrule and \bottomrule in tables

%----------------------------------------------------------------------------------------
%	TÍTULO
%----------------------------------------------------------------------------------------

%% Título y otros %%
\title[Historia de las Matemáticas en China e India]{\Large Historia de las Matemáticas en China e India\\
\normalsize Aritmética, Resolución de Ecuaciones y Sistemas de Numeración} % The short title appears at the bottom of every slide, the full title is only on the title page

\author[Óscar Bermúdez, Sonia González, Alba Mª Toledo]{
	Óscar Bermúdez Garrido, \href{http://www.github.com/oxcar103}{\textcolor{violet}{@oxcar103}}\\
	Sonia González Sánchez \\
	Alba María Toledo Pérez
} % Your name

\institute[UGR] % Your institution as it will appear on the bottom of every slide, may be shorthand to save space
{
  Universidad de Granada \\ % Your institution for the title page
}
\date{\today} % Date, can be changed to a custom date

\begin{document}

% Diapositiva de título.
\begin{frame}
	\titlepage % Print the title page as the first slide
\end{frame}


%----------------------------------------------------------------------------------------
%	PRESENTACIÓN
%----------------------------------------------------------------------------------------
 
%------------------------------------------------


\section{Desarrollo de las Matemáticas en China}
	\subsection{Sistema de numeración chino y aritmética}
		\begin{frame}
			\frametitle{Sistema de numeración chino y aritmética}
			\begin{figure}
			\centering
				\begin{subfigure}{.5\textwidth}
					\centering
					\includegraphics[width = .5\linewidth]{chinese_numbers.jpg}
					\caption{Caracteres chinos utilizados en el sistema de numeración.}
				\end{subfigure}%
				\begin{subfigure}{.5\textwidth}
					\centering
					\includegraphics[width = .5\linewidth]{Chinese-abacus.jpg}
					\caption{Ábaco \textit{suan pan} utilizado en China desde la Antigüedad.}
				\end{subfigure}
			\end{figure}
		\end{frame}

	\subsection{Jiuzhang Suanshu o Nueve Capítulos sobre el Arte Matemático}
		\begin{frame}
			\frametitle{\textit{Jiuzhang Suanshu} o \textit{Nueve Capítulos sobre el Arte Matemático}}
			Texto
		\end{frame}

\section{Matemáticos destacados de China}
	\subsection{Liu Hui (220 - 280)}
		\begin{frame}
			\only<1>{\frametitle{Liu Hui (220 - 280)}
			\begin{figure}
				\centering
				\includegraphics[width = .25\linewidth]{liu_hui.jpg}
				\caption{Lui Hui(220 - 280).}
			\end{figure}}
		\end{frame}

	\subsection{Zu Chongzhi (429 - 501)}
		\begin{frame}
			\only<1>{\frametitle{Zu Chongzhi (429 - 501)}
			\begin{figure}
				\centering
				\includegraphics[width = .25\linewidth]{zu_chongzhi.jpg}
				\caption{Zu Chongzhi (429 - 501).}
			\end{figure}}
		\end{frame}

	\subsection{Qin Jiushao (1202 - 1261)}
		\begin{frame}
			\only<1>{\frametitle{Qin Jiushao (1202 - 1261)}
			\begin{figure}
				\centering
				\includegraphics[width = .4\linewidth]{qin_jiushao.jpg}
				\caption{Qin Jiushao (1202 - 1261).}
			\end{figure}}
		\end{frame}

	\subsection{Shiing-Shen Chern (1911 - 2004)}
		\begin{frame}
			\only<1>{\frametitle{Shiing-Shen Chern (1911 - 2004)}
			\begin{figure}
				\centering
				\includegraphics[width = .25\linewidth]{shiing-shen_chern.jpg}
				\caption{Shiing-Shen Chern(1911 - 2004).}
			\end{figure}}
		\end{frame}

	\subsection{Chen Jingrun (1933 - 1996)}
		\begin{frame}
			\only<1>{\frametitle{Chen Jingrun (1933 - 1996)}
			\begin{figure}
				\centering
				\includegraphics[width = .25\linewidth]{chen_jingrun.jpg}
				\caption{Chen Jingrun (1933 - 1996).}
			\end{figure}}
		\end{frame}

\section{Desarrollo de las Matemáticas en la India}
	\subsection{Sistema de numeración indio y aritmética}
		\begin{frame}
			\only<1>{\frametitle{Sistema de numeración chino y aritmética}

			\begin{figure}
				\centering
				\includegraphics[width = .3\linewidth]{indian_numbers.jpg}
				\caption{Símbolos utilizados en el sistema de numeración indio.}
			\end{figure}}

			\only<2>{\begin{figure}
				\centering
				\includegraphics[width = .4\linewidth]{indian_multiplication.png}
				\caption{Multiplicación en celosía.}
			\end{figure}}

			\only<3>{\begin{figure}
			\centering
				\begin{subfigure}{.5\textwidth}
					\centering
					\includegraphics[width = .75\linewidth]{Galley_Method3.png}
					\caption{División paso a paso.}
				\end{subfigure}%
				\begin{subfigure}{.5\textwidth}
					\centering
					\includegraphics[width = .5\linewidth]{Galley_Method4.png}
					\caption{Solución final.}
				\end{subfigure}
				\caption{División en Galera.}
			\end{figure}}
			
		\end{frame}

\section{Matemáticas destacados de la India}
	\subsection{Aryabhata (476 - 550)}
		\begin{frame}
			\only<1>{\frametitle{Aryabhata (476 - 550)}
			\begin{figure}
				\centering
				\includegraphics[width = .4\linewidth]{aryabhata.jpg}
				\caption{Aryabhata (476 - 550).}
			\end{figure}}
		\end{frame}

	\subsection{Varahamihira (505 - 587)}
		\begin{frame}
			\only<1>{\frametitle{Varahamihira (505 - 587)}
			\begin{figure}
				\centering
				\includegraphics[width = .3\linewidth]{varahamihira.jpg}
				\caption{Varahamihira (505 - 587).}
			\end{figure}}
		\end{frame}

	\subsection{Brahmagupta (598 - 660)}
		\begin{frame}
			\only<1>{\frametitle{Brahmagupta (598 - 660)}
			\begin{figure}
				\centering
				\includegraphics[width = .3\linewidth]{brahmagupta.jpg}
				\caption{Brahmagupta (598 - 660).}
			\end{figure}}
		\end{frame}

	\subsection{Bhaskara II (1114 - 1185)}
		\begin{frame}
			\only<1>{\frametitle{Bhaskara II (1114 - 1185)}
			\begin{figure}
				\centering
				\includegraphics[width = .3\linewidth]{bhaskara_II.jpg}
				\caption{Bhaskara II (1114 - 1185).}
			\end{figure}}
		\end{frame}
		
	\subsection{Madhava (1350 - 1425)}
		\begin{frame}
			\only<1>{\frametitle{Madhava (1350 - 1425)}
			\begin{figure}
				\centering
				\includegraphics[width = .35\linewidth]{madhava.jpg}
				\caption{Madhava (1340 - 1425).}
			\end{figure}}
		\end{frame}

\section{Referencias}
	% Bibliografía
	\begin{frame}
		\frametitle{Referencias}
		
		Para la realización de estas diapositivas, se han utilizado los repositorios de GitHub:
		
		\footnotesize{
		  \begin{thebibliography}{7} % Beamer does not support BibTeX so references must be inserted manually as below
		    \bibitem{Pbaeyens} Pablo Baeyens
		      \newblock Guía de uso de beamer
		      \newblock \url{https://github.com/dgiim/beamer}
		      
		    \bibitem{M42} Mario Román
		      \newblock Recopilación de plantillas de Latex.
		      \newblock \url{https://github.com/M42/plantillas}
		  \end{thebibliography}
		}
	\end{frame}

%------------------------------------------------

\begin{frame}
\Huge{\centerline{Fin.}}
\end{frame}

%----------------------------------------------------------------------------------------

\end{document}
