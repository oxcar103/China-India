%%
% Plantilla de Presentación
% Modificación de una plantilla de Latex de LaTeXTemplates para adaptarla 
% al castellano y a las necesidades de escribir informática y matemáticas.
%
% Editada por: Mario Román
%
% License:
% CC BY-NC-SA 3.0 (http://creativecommons.org/licenses/by-nc-sa/3.0/)
%%

%%%%%%%%%%%%%%%%%%%%%
% Beamer Presentation
% LaTeX Template
% Version 1.0 (10/11/12)
%
% This template has been downloaded from:
% http://www.LaTeXTemplates.com
%
% License:
% CC BY-NC-SA 3.0 (http://creativecommons.org/licenses/by-nc-sa/3.0/)
%
%%%%%%%%%%%%%%%%%%%%%

%----------------------------------------------------------------------------------------
%	PAQUETES Y CONFIGURACIÓN DEL DOCUMENTO
%----------------------------------------------------------------------------------------

\documentclass[12pt, aspectratio=169]{beamer} % Beamer
\usepackage[spanish]{babel} % Traducciones
\usepackage[utf8]{inputenc} % Uso de caracteres UTF-8
\usepackage[T1]{fontenc} % Permite copiar código y evita errores
\uselanguage{Spanish} % Traducciones beamer
\languagepath{Spanish} % (tex.stackexchange.com/questions/168208)
\usepackage{pgfpages} % Beamer User Guide sections 19.6 and 22
\usepackage[absolute,overlay]{textpos} % Especifica posición del texto.

%% Temas %%
% Tema y tema de color
\usetheme{Dresden}
\usecolortheme{dolphin}

% Fuentes de tamaño arbitrario
\usepackage{lmodern}

% Gráficos
\usepackage{caption}
\usepackage{subcaption} % Allows including 2 images side to side
\usepackage{graphicx} % Allows including images
\usepackage{booktabs} % Allows the use of \toprule, \midrule and \bottomrule in tables

%----------------------------------------------------------------------------------------
%	TÍTULO
%----------------------------------------------------------------------------------------

%% Título y otros %%
\title[Historia de las Matemáticas en China e India]{\Large Historia de las Matemáticas en China e India\\
\normalsize Aritmética, Resolución de Ecuaciones y Sistemas de Numeración} % The short title appears at the bottom of every slide, the full title is only on the title page

\author[Óscar Bermúdez, Sonia González, Alba Mª Toledo]{
	Óscar Bermúdez Garrido, \href{http://www.github.com/oxcar103}{\textcolor{violet}{@oxcar103}}\\
	Sonia González Sánchez \\
	Alba María Toledo Pérez
} % Your name

\institute[UGR] % Your institution as it will appear on the bottom of every slide, may be shorthand to save space
{
  Universidad de Granada \\ % Your institution for the title page
}
\date{\today} % Date, can be changed to a custom date

\begin{document}

% Diapositiva de título.
\begin{frame}
	\titlepage % Print the title page as the first slide
\end{frame}


%----------------------------------------------------------------------------------------
%	PRESENTACIÓN
%----------------------------------------------------------------------------------------
 
%------------------------------------------------


\section{Desarrollo de las Matemáticas en China}
	\begin{frame}
		Dadas las diferencias lingüísticas y geográficas, el desarrollo de las matemáticas chinas se pueden considerar
		independientes de las culturas occidentales hasta el siglo XVII.
		
		\pause
		
		El desarrollo de las matemáticas se centró en:
		\begin{itemize}
			\item La astronomía para perfeccionar el calendario agrícola.
			\item Los negocios.
			\item La medida de las tierras.
			\item Los impuestos.
		\end{itemize}
	\end{frame}

	\subsection{Sistema de numeración chino y aritmética}
		\begin{frame}
			\frametitle{Sistema de numeración chino y aritmética}
			
			\only<1>{El sistema de numeración chino usa caracteres que denotan las unidades del 1 al 9 y las potencias de 10.
			En él, las cifras de las unidades multiplican a las cifras de las potencias de 10 y el resultado se suma.
			
			En este sistema no se necesitaba cifra para el cero.}
			
			\begin{figure}
			\centering
				\begin{subfigure}{.5\textwidth}
					\centering
					\includegraphics[width = .5\linewidth]{chinese_numbers.jpg}
					\caption{Caracteres chinos.}
				\end{subfigure}%
				\pause
				\begin{subfigure}{.5\textwidth}
					\centering
					\includegraphics[width = .5\linewidth]{Chinese-abacus.jpg}
					\caption{Ábaco \textit{suan pan}.}
				\end{subfigure}
			\end{figure}
			
			Para las operaciones habituales, se usaba el ábaco chino o \textit{suan pan}. Cada columna es una
			potencia de 10, con las unidades a la derecha, y disponen de 2 cuentas de cielo y 5 cuentas de tierra.
			
			\pause
			
			\alert{La cuenta de cielo extra se utilizaba para facilitar la multiplicación y división.}
			
		\end{frame}

	\subsection{Jiuzhang Suanshu o Nueve Capítulos sobre el Arte Matemático}
		\begin{frame}
			\frametitle{\textit{Jiuzhang Suanshu} o \textit{Nueve Capítulos sobre el Arte Matemático}}
			\only<1>{\textit{Jiuzhang Suanshu} es un libro de matemáticas chino anónimo de gran importancia en Oriente, hasta
			el punto de tener similar importancia que \textit{Los Elementos de Euclides}.
			
			Esta obra se centra en cómo resolver problemas en contraparte con los griegos, que deducían proposiciones a
			partir de axiomas.}
			
			\only<2>{El contenido de \textit{Jiuzhang Suanshu} es:
			\begin{itemize}
				\item \textit{Fangtian} - Campos delimitadores.
				\item \textit{Sumi} - Mijo y arroz. Intercambio y precios
				\item \textit{Cuifen} - Distribución proporcional.
				\item \textit{Shaoguang} - Reducción de dimensiones.
				\item \textit{Shanggong} - Para la construcción. Volúmenes.
				\item \textit{Junshu} - Impuestos equitativos. Proporción.
				\item \textit{Yingbuzu} - Exceso y déficit. La regla de la falsa posición.
				\item \textit{Fangcheng} - Sistemas de ecuaciones. Eliminación gaussiana.
				\item \textit{Gougu} - Base y altitud. Teorema de Gougu(Pitágoras).
			\end{itemize}}
			
		\end{frame}

\section{Matemáticos destacados de China}
	\subsection{Liu Hui (220 - 280)}
		\begin{frame}
			\frametitle{Liu Hui (220 - 280)}
			\begin{figure}
				\centering
				\includegraphics[width = .25\linewidth]{liu_hui.jpg}
				\caption{Lui Hui(220 - 280).}
			\end{figure}
		\end{frame}
		
		\begin{frame}
			Liu Hui destaca por realizar un comentario explicando el \textit{Jiuzhang Suanshu}:

			\begin{itemize}
				\item Explica una demostración al teorema de Gougu(Pitágoras).
				\item Hay un algoritmo para el cálculo de $\pi$ y obtuvo $\pi = 3.1416$.
				\item Resuelve sistemas de ecuaciones lineales usando eliminación de Gauss.
				\item Calcula el volumen de varias figuras de la geometría espacial.
			\end{itemize}
	
			\pause

			También escribió el \textit{Haidao suajing} o \textit{Manual Matemático de la Isla Marina}, que es un manual
			práctico para resolver problemas cotidianos:
			
			\begin{itemize}
				\item La altura de un árbol en una colina.
				\item La profundidad de un barranco.
				\item El tamaño de una ciudad vista desde una montaña.
			\end{itemize}	
		\end{frame}

	\subsection{Zu Chongzhi (429 - 501)}
		\begin{frame}
			\frametitle{Zu Chongzhi (429 - 501)}
			\begin{figure}
				\centering
				\includegraphics[width = .25\linewidth]{zu_chongzhi.jpg}
				\caption{Zu Chongzhi (429 - 501).}
			\end{figure}
		\end{frame}
					
		\begin{frame}
			Zu Chongzhi destacó por su talento desde temprana edad.
			
			Sus logros fueron:
			\pause
			\begin{itemize}
				\item Calcular un año como 365.24281481($\approx$ 365.24219878) días.
				\item Predecir con éxito cuatro eclipses.
				\pause
				\item La fracción de Zu $\displaystyle \left(\frac{355}{113}\right)$ como aproximación a $\pi$ y
				resultó ser la aproximación racional más cercana a $\pi$ con denominador inferior a 16600.
				\pause
				\item Deducir el volumen de una esfera como $\frac{\pi D^3}{6}$ donde $D$ es diámetro para Oriente,
				ya que no llegó el descubrimiento de Arquímedes hasta China.
			\end{itemize}
			
			\pause
			
			Escribió sus logros en el \textit{Zhui Shu} o \textit{Métodos para la Interpolación} pero eran muy
			adelantados para su época, resultaron confusos y fueron descartados, por lo que se perdieron.
		\end{frame}

	\subsection{Qin Jiushao (1202 - 1261)}
		\begin{frame}
			\frametitle{Qin Jiushao (1202 - 1261)}
			\begin{figure}
				\centering
				\includegraphics[width = .4\linewidth]{qin_jiushao.jpg}
				\caption{Qin Jiushao (1202 - 1261).}
			\end{figure}
		\end{frame}			

		\begin{frame}
			Qin Jiushao es considerado como uno de los más grandes matemáticos y \only<1>{\alert{no dedicó su vida a las
			matemáticas}} \only<2->{no dedicó su vida a las matemáticas}.
			
			\pause 
			
			Qin era experto en esgrima, música, arquitectura,... y tuvo muchos puestos administrativos y burocráticos.
					
			Como funcionario del gobierno era corrupto, y fue acusado de soborno y de envenenar a sus enemigos...
			\only<3>{Un cafre.}
			
			\only<4>{Estudió el \textit{Jiuzhang Suanshu} y escribió el \textit{Shushu Jiuzhang} o \textit{Tratado
			Matemático en Nueve Secciones} siguiendo el mismo modelo.
			
			Este libro es el primer ejemplar que utiliza \textit{0} para el cero.}
		\end{frame}

	\subsection{Shiing-Shen Chern (1911 - 2004)}
		\begin{frame}
			\frametitle{Shiing-Shen Chern (1911 - 2004)}
			\begin{figure}
				\centering
				\includegraphics[width = .25\linewidth]{shiing-shen_chern.jpg}
				\caption{Shiing-Shen Chern(1911 - 2004).}
			\end{figure}
		\end{frame}
		
		\begin{frame}
			\begin{itemize}
				\item Trabajó en el Instituto de Estudios Avanzados (IAS) de Princeton.
				\item Fue profesor en la Universidad de Chicago.
				\item Perteneció a la Academia Nacional de Ciencias de los Estados Unidos.
				\item Fue vicepresidente de la American Mathematical Society (AMS).
				\item Fundó el Instituto de Matemáticas de la Academia Sínica.
				\item Fundó el Instituto de Investigación de las Ciencias Matemáticas (MSRI) o Instituto de Chern de
				Matemáticas tras su muerte.
			\end{itemize}
			
			\pause
			
			El trabajo de Chern abarca los campos clásicos de:
			\begin{itemize}
				\item La teoría Chern-Simons.
				\item La teoría Chern-Weil.
				\item Las clases de Chern.
				\item La geometría diferencial proyectiva y redes matemáticas.
			\end{itemize}
		\end{frame}	

		\begin{frame}
			Shiing-Shen Chern es considerado uno de los mayores matemáticos del siglo XX.
			
			En su honor, cada 4 años el Congreso Internacional de Matemáticos y la Unión Matemática Internacional entregan
			la Medalla Chern, que premia la dedicación de toda una vida de trabajo a las matemáticas en su nivel más alto.
		\end{frame}
		
	\subsection{Chen Jingrun (1933 - 1996)}
		\begin{frame}
			\frametitle{Chen Jingrun (1933 - 1996)}
			\begin{figure}
				\centering
				\includegraphics[width = .25\linewidth]{chen_jingrun.jpg}
				\caption{Chen Jingrun (1933 - 1996).}
			\end{figure}
			
		\end{frame}
		\begin{frame}
			Chen Jingrun realizó contribuciones significativas a la teoría numérica.
			
			Fue un ídolo nacional en la década de los setenta.
			
			Trabajó en:
			\begin{itemize}
				\item La conjetura principal gemela.
				\item El problema de Waring.
				\item La conjetura de Legendre.
				\item La conjetura de Goldbach.
			\end{itemize}
		\end{frame}
		
		\begin{frame}
			En 1966, demostró una de las mejores aproximaciones a la conjetura de Goldbach, el \textit{Teorema de Chen}:
			
			\begin{center}
				\em
				Todo entero par suficientemente grande se puede escribir como la suma de un número primo y de un semiprimo
				o casiprimo, es decir, un entero que es el producto de, a lo sumo, dos números primos.
			\end{center}						
		\end{frame}

\section{Desarrollo de las Matemáticas en la India}
	\begin{frame}
		La matemática india se desarrolló, al igual que China, prácticamente independiente del desarrollo matemático
		occidental, dando preferencia al cálculo numérico frente al rigor deductivo.

		Inicialmente, las obras indias se transmitían de forma oral, por ello se solían componer de una parte en verso y
		una parte en prosa.
		
		\pause
		
		Se puede apreciar dos épocas de desarrollo en las matemáticas indias:
		\begin{itemize}
			\item La época clásica(siglos I al VIII): Aryabhata, Varahamihira, Brahmagupta, Bhaskara I, Mahavira,
			Bakhshali, Bhaskara II.
			\item La escuela de Kerala(1300- 1600): se realizaron grandes desarrollos matemáticos, aunque desgraciadamente
			no parece haber indicios de difusión fuera de la escuela. Destacan Madhava de Sangamagrama y Nilakantha Somayaji.
		\end{itemize}
	\end{frame}
	\begin{frame}
		Entre sus mayores logros, se pueden encontrar:
		\begin{itemize}
			\item El concepto de cero como número.
			\item Los números negativos para representar deudas.
			\item El planteamiento algebraico de ecuaciones de primer y segundo grado.
			\item Ternas pitagóricas.
			\item Raíces cuadradas.
			\item El desarrollo de conceptos básicos de la trigonometría.
			\item El desarrollo de las expansiones en series.
		\end{itemize}
	\end{frame}

	\subsection{Sistema de numeración indio y aritmética}
		\begin{frame}
			\only<1>{\frametitle{Sistema de numeración indio y aritmética}}
		
			La mayor característica y, sin duda, la más conocida del sistema de numeración india es el uso de un sistema
			posicional de base 10 con cero.
			
			Además, con una grafía muy similar a la actual.
			
			\only<2>{\begin{figure}
				\centering
				\includegraphics[width = .3\linewidth]{indian_numbers.jpg}
				\caption{Símbolos utilizados en el sistema de numeración indio.}
			\end{figure}}
			
			\only<3>{Quizás la característica más distintiva de este sistema frente al actual sea tener un $"$.$"$ en la
			posición de las centenas, en lugar de los millares.
			
			Un millón: \textit{1.00.00.00}}.
		\end{frame}
			
		\begin{frame}
			\begin{figure}
				\centering
				\includegraphics[width = .7\linewidth]{indian_multiplication.png}
				\caption{Multiplicación en celosía.}
			\end{figure}
		\end{frame}	
				
		\begin{frame}
			\begin{figure}
			\centering
				\begin{subfigure}{.5\textwidth}
					\centering
					\includegraphics[width = .75\linewidth]{Galley_Method3.png}
					\caption{División paso a paso.}
				\end{subfigure}%
				\begin{subfigure}{.5\textwidth}
					\centering
					\includegraphics[width = .5\linewidth]{Galley_Method4.png}
					\caption{Solución final.}
				\end{subfigure}
				\caption{División en Galera.}
			\end{figure}
			
			\pause
			
			El nombre deriva del parecido gráfico que se genera con este método y una galera.
		\end{frame}

\section{Matemáticas destacados de la India}
	\subsection{Aryabhata (476 - 550)}
		\begin{frame}
			\frametitle{Aryabhata (476 - 550)}
			\begin{figure}
				\centering
				\includegraphics[width = .4\linewidth]{aryabhata.jpg}
				\caption{Aryabhata (476 - 550).}
			\end{figure}
		\end{frame}
		
		\begin{frame}
			Se deduce cuándo nació por su principal obra, \textit{Aryabhatiya}, y es el escritor de álgebra más antiguo
			de quien se tiene conocimiento.
			
			En su obra, Aryabhata destacó por:
			\begin{itemize}
				\item Aproximar el número $\pi$ (3,1416) y se especula que indica que el valor es inconmensurable.
				\item Ser el primero en encontrar el radio de la Tierra y el único en calcular su volumen.
				\item No usa el cero, pero se cree que aparece implícito.					
				\item Dar la suma de series matemáticas de cuadrados y cubos:
				\begin{itemize}
					\item $\displaystyle 1^2 + 2^2 + \cdots + n^2 = \frac{n(n + 1)(2n + 1)}{6}$
					\item $\displaystyle 1^{3} + 2^{3} + \cdots + n^{3} = (1+2+\cdots +n)^{2}$ 
				\end{itemize}
			\end{itemize}			
		\end{frame}
			
		\begin{frame}
			\textit{Aryabhatiya} tuvo una gran influencia en la matemática de la India pero por su estructura sólo ha
			podido ser descifrada a través de comentaristas como su discípulo Bhaskara I o Nilakantha Somayaji.
								
						
			\textit{Aryabhatiya} consiste en 108 versos y 13 versos introductorios:
			\begin{itemize}
				\item \textit{Gitikapada} (13 versos):  Tratan grandes unidades de tiempo.
				\item \textit{Ganitapada} (33 versos): Trata aritmética, progresiones geométricas y ecuaciones.
				\item \textit{Kalakriyapada} (25 versos): Trata diferentes unidades de tiempo y los planetas.
				\item \textit{Golapada} (50 versos): Trata aspectos geométricos/trigonométricos de la Tierra.
			\end{itemize}
		\end{frame}

	\subsection{Varahamihira (505 - 587)}
		\begin{frame}
			\frametitle{Varahamihira (505 - 587)}
			\begin{figure}
				\centering
				\includegraphics[width = .3\linewidth]{varahamihira.jpg}
				\caption{Varahamihira (505 - 587).}
			\end{figure}
		\end{frame}

		\begin{frame}
			Trabajó en la escuela de matemáticas en Ujjain.
			
			Sus grandes descubrimientos matemáticos fueron:
			\begin{itemize}
				\item Encontró fórmulas trigonométricas equivalentes a:
				\begin{itemize}
					\item $\displaystyle \sin x = \cos \left(\frac{\pi}{2} - x \right)$
					\item $\displaystyle \sin^2 x + \cos^2 x = 1$
					\item $\displaystyle \frac{1 - \cos(2x)}{2} = \sin^2 x$
				\end{itemize}
				\item Obtuner tablas de senos muy precisas.
				\item Trabajar con cuadrados mágicos.
				\item Calcular $_n C_r$ usando el triángulo de Pascal.
			\end{itemize}
		\end{frame}

		\begin{frame}
			Su obra más famosa, \textit{Pancasiddhantika}(\textit{Los Cinco Cánones Astronómicos}), resume cinco tratados
			astronómicos anteriores que se han perdido:
			\begin{enumerate}
				\item \textit{Surya siddhanta}.
				\item \textit{Romaka siddhanta}.
				\item \textit{Paulisa siddhanta}.
				\item \textit{Vasishtha siddhanta}.
				\item \textit{Paitamaha siddhanta}.
			\end{enumerate}
		\end{frame}
		
	\subsection{Brahmagupta (598 - 660)}
		\begin{frame}
			\only<1>{\frametitle{Brahmagupta (598 - 660)}
			\begin{figure}
				\centering
				\includegraphics[width = .3\linewidth]{brahmagupta.jpg}
				\caption{Brahmagupta (598 - 660).}
			\end{figure}}
		\end{frame}

	\subsection{Bhaskara II (1114 - 1185)}
		\begin{frame}
			\only<1>{\frametitle{Bhaskara II (1114 - 1185)}
			\begin{figure}
				\centering
				\includegraphics[width = .3\linewidth]{bhaskara_II.jpg}
				\caption{Bhaskara II (1114 - 1185).}
			\end{figure}}
		\end{frame}
		
	\subsection{Madhava (1350 - 1425)}
		\begin{frame}
			\only<1>{\frametitle{Madhava (1350 - 1425)}
			\begin{figure}
				\centering
				\includegraphics[width = .35\linewidth]{madhava.jpg}
				\caption{Madhava (1340 - 1425).}
			\end{figure}}
		\end{frame}





\section{Introducción}
	\begin{frame}
		Para webs:
		
		\begin{itemize}
			\item \uncover<2->{Control de seguridad}
			
%			\only<3>{\begin{figure}
%				\includegraphics[width=5cm]{Thief.jpg}
%			\end{figure}}
%
%			\item \uncover<4->{Gestión de concurrencia}
%			
%			\only<5>{\begin{figure}
%				\includegraphics[width=10cm]{Atasco.jpg}
%			\end{figure}}
%			
%			\item \uncover<6->{Manejo de peticiones de clientes}
%			
%			\only<7>{\begin{figure}
%				\includegraphics[width=10cm]{Burocracia.jpg}
%			\end{figure}}

		\end{itemize}

		\uncover<8->{Esto provoca que el desarrollador haga un esfuerzo adicional que distrae la atención
		sobre la lógica de negocio.}
	\end{frame}

\section{Java EE}
	\begin{frame}
		Hay varias.
		
		\pause
		
		\alert{Java.}
	\end{frame}

\section{Presentación de los programas}
	\subsection{Apache Tomcat}
		\begin{frame}
			\frametitle{Apache Tomcat}
			
%			\begin{figure}
%				\includegraphics[width=3cm]{Tomcat_logo.jpg}
%			\end{figure} 
			
			\pause
			
			Apache.
		\end{frame}

\section{Comparación}
	\begin{frame}
		\frametitle{Resultados}
	
		\only<1>{Se agrupan los resultados en tablas y se calcula qué servicio es mejor en términos de...}

		\only<2>{hola}
%		\begin{figure}
%			Tiempo:
%			
%			\includegraphics[width=10cm]{Comparativa_tiempo.png}
%		\end{figure}}
%
%		\only<3>{\begin{figure}
%			Uso de CPU:
%			
%			\includegraphics[width=10cm]{Comparativa_CPU.png}
%		\end{figure}}
%
%		\only<4>{\begin{figure}
%			Uso de RAM:
%			
%			\includegraphics[width=13cm]{Comparativa_RAM.png}
%		\end{figure}}
	
	\end{frame}

\section{Referencias}
	% Bibliografía
	\begin{frame}
		\frametitle{Referencias}
		
		Para la realización de estas diapositivas, se han utilizado los repositorios de GitHub:
		
		\footnotesize{
		  \begin{thebibliography}{7} % Beamer does not support BibTeX so references must be inserted manually as below
		    \bibitem{Pbaeyens} Pablo Baeyens
		      \newblock Guía de uso de beamer
		      \newblock \url{https://github.com/dgiim/beamer}
		      
		    \bibitem{M42} Mario Román
		      \newblock Recopilación de plantillas de Latex.
		      \newblock \url{https://github.com/M42/plantillas}
		  \end{thebibliography}
		}
	\end{frame}

%------------------------------------------------

\begin{frame}
\Huge{\centerline{Fin.}}
\end{frame}

%----------------------------------------------------------------------------------------

\end{document}
